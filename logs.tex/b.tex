\begin{poem}{Б}{~}
	\addcontentsline{toc}{section}{Б}
	\begin{altverse}
	Этот бал безупречно бел,\\
	Этот бог беспардонно бит,\\
	Жизнь до станции "Англетер"\\
	не дойдет,\\
	так повременит.\\
	Этот миг слишком мил и мал,\\
	Этот вздох слишком волен был,\\
	Этот бог --- безупречный бал,\\
	Но настолько бел, что убил.\\
	Строим стенки для нор и крыс ---\\
	Население прет наверх.\\
	Этот бал беспардонно тих ---\\
	Это бог обесцветил век.\\
	Это словно удар с плеча,\\
	Наши выкрики не звучат:\\
	Это слишком зеленый чай,\\
	Это слишком приятный чад.\\
	Этот взгляд от души до души\\
	(Что за маску ты снова надел?),\\
	Этот бог никуда не спешит ---\\
	Этот бал априори бел.\\
	Ты молчишь у стихов между строк,\\
	Усмехаешься, глядя в зал:\\
	Ну зачем тебе этот бог,\\
	Ну зачем тебе этот бал?\\
	Мы придумали мир с нуля ---\\
	Этот мир мало-мальски мил,\\
	Этот бал до конца обеля,\\
	Этот бог безупречно был.\\
	Ты шептала:\\
	"Прости-прощай,\\
	Ну к чему этот новый век?\\
	Этот вечнозеленый чай,\\
	Этот вечноленивый бег,\\
	Этот скорбный \\
	(сквозь слезы)\\
	смех,\\
	Этот запах крысиных нор?.."\\
	И все это ты шепчешь вверх\\
	Белоснежному богу в упор.\\
	Я скажу тебе:\\
	"Просто так,\\
	Чтобы было чего спросить.\\
	Этот бог обернется в флаг ---\\
	Христарадничать по Руси.\\
	Рассмеешься и будешь петь,\\
	Чтобы ни было дальше --- будь,\\
	Чтобы бал продолжал белеть,\\
	И богам не давал уснуть,\\
	Чтобы вечно глаза в глаза\\
	И при этом --- не видя лиц,\\
	Чтобы вроде бы все сказать,\\
	За безумно короткий блиц,\\
	Чтобы все это ---\\
	здесь и сейчас,\\
	Все, кто слушает нас, поймут.\\
	В чашках снова остынет чай,\\
	Нам опять не хватит минут,\\
	И мы заново будем жить\\
	И напишем на стенах нор,\\
	Будто мы обещали \\
	быть\\
	До вот тех невозможных пор.\\
	И в попытке найти изъян\\
	В мельтешенье ненаших тел,\\
	Будет бог безупречно пьян,\\
	Будет бал беспардонно бел".\\
	\end{altverse}
\end{poem}


%-------------------------------------------%
